% Solution design chapter % 

\begin{chapterabstract}
% put abstract here %
\end{chapterabstract}

\section{Technologies}

\subsection{Library \texttt{devtools}}

The \texttt{devtools} package streamlines R package development by providing functions to create package skeletons, load code, run tests, and build documentation in a single workflow \cite{wickham2019devtools}. It integrates tightly with \texttt{roxygen2} for in‐source documentation and with continuous integration services.

\begin{verbatim}
# Create a new package skeleton
library(devtools)
create("MyPackage")

# Load all R code without installing
load_all("MyPackage/")

# Run tests in tests/testthat/
test("MyPackage")

# Document and build package
document("MyPackage/")
build("MyPackage/")
\end{verbatim}

By automating these repetitive tasks, \texttt{devtools} reduces boilerplate and accelerates iterative development.

\subsection{Library \texttt{jsonlite}}

The \texttt{jsonlite} package provides a robust, high‐performance JSON parser and generator for R, supporting both streaming and in‐memory operations. It maps R objects to JSON with consistent type coercion rules, making it ideal for API interaction and data interchange \cite{ooms2014jsonlite}.

\begin{verbatim}
library(jsonlite)

# Parse JSON string into R list/data.frame
json_data <- fromJSON('{"x":1,"y":[2,3],"z":{"a":true}}')

# Convert R object back to pretty JSON
json_pretty <- toJSON(json_data, pretty = TRUE)
cat(json_pretty)
\end{verbatim}

This two-way mapping ensures round-trip fidelity when exchanging data with web services.

\subsection{Library \texttt{httr}}

\texttt{httr} simplifies working with HTTP from R by wrapping low‐level HTTP methods in user-friendly functions. It handles authentication, redirection, and cookies, and parses responses into R objects \cite{wickham2011httr}.

\begin{verbatim}
library(httr)

# Basic GET request
resp <- GET("https://api.example.com/data",
            add_headers(Accept = "application/json"))

# Check for HTTP errors
stop_for_status(resp)

# Parse JSON content
data <- content(resp, as = "parsed")
\end{verbatim}

For POST requests with JSON payloads, \texttt{httr} manages encoding and content‐type headers automatically.

\subsection{Library \texttt{furrr}}

The \texttt{furrr} package combines the mapping functions of \texttt{purrr} with the parallel execution capabilities of the \texttt{future} framework, allowing seamless parallel iteration in R \cite{Vaughan2020furrr}.

\begin{verbatim}
library(furrr)
plan(multisession, workers = 4)

# Parallel map over a vector
results <- future_map(1:10, function(x) {
  Sys.sleep(1)  # simulate work
  x^2
})
\end{verbatim}

By changing the \texttt{plan()}, you can switch between multicore, multisession, or cluster‐based execution with minimal code changes.

\subsection{GoogleTest}

GoogleTest is Google’s C++ testing framework offering rich assertions, test fixtures, and death tests for verifying code correctness. Its expressive macros and automatic test discovery facilitate maintainable C++ test suites \cite{google2023gtest}.

\begin{verbatim}
// example.cpp
#include <gtest/gtest.h>

// Function under test
int square(int x) {
  return x * x;
}

// Test case
TEST(SquareTest, HandlesPositive) {
  EXPECT_EQ(square(3), 9);
  EXPECT_NE(square(3), 8);
}

int main(int argc, char **argv) {
  ::testing::InitGoogleTest(&argc, argv);
  return RUN_ALL_TESTS();
}
\end{verbatim}

Compile and run with:
\begin{verbatim}
g++ example.cpp -lgtest -pthread -o example_test
./example_test
\end{verbatim}

\subsection{OpenAI API}

The OpenAI API provides RESTful endpoints for interacting with language models. Using \texttt{httr} in R, you can send requests and handle responses for completions, embeddings, and more \cite{openai2023api}.

\begin{verbatim}
library(httr)
library(jsonlite)

api_key <- Sys.getenv("OPENAI_API_KEY")

resp <- POST(
  "https://api.openai.com/v1/completions",
  add_headers(Authorization = paste("Bearer", api_key)),
  content_type_json(),
  encode = "json",
  body = list(
    model       = "text-davinci-003",
    prompt      = "Summarize the benefits of mutation testing:",
    max_tokens  = 100,
    temperature = 0.7
  )
)

stop_for_status(resp)
result <- content(resp, as = "parsed")
cat(result$choices[[1]]$text)
\end{verbatim}

This integration enables programmatic access to advanced NLP capabilities directly from R.


\section{Design constraint}

Designing a mutation testing tool for the R language requires careful consideration of multiple constraints to ensure usability, maintainability, and performance in production environments.

\subsection{Customer Requirements}
The tool must cater to R users—ranging from data scientists to statisticians—by providing:
\begin{itemize}
  \item \textbf{Seamless Integration:} A familiar API that builds on \texttt{testthat} conventions, minimizing learning curve \cite{wickham2011testthat}.
  \item \textbf{Ease of Installation:} Simple installation via CRAN or \texttt{devtools::install\_github()}, with minimal external dependencies \cite{wickham2019devtools}.
  \item \textbf{Clear Reporting:} Human‐readable mutation score summaries and detailed mutant tracebacks for straightforward interpretation.
\end{itemize}

\subsection{Business Needs}
From an organizational standpoint, the tool must:
\begin{itemize}
  \item \textbf{Cost Efficiency:} Leverage existing infrastructure (R, CI/CD) to avoid additional licensing or hardware costs.
  \item \textbf{Return on Investment:} Provide actionable insights that improve test suite quality and reduce production defects, as shown by improved test robustness metrics in industry case studies \cite{petrovic2018industrial}.
  \item \textbf{Adoption and Support:} Comply with CRAN policies and support major R versions (≥3.6) to maximize user adoption and community contributions \cite{R-base}.
\end{itemize}

\subsection{Performance Constraints}
Mutation testing is inherently compute‐intensive, so the tool must:
\begin{itemize}
  \item \textbf{Parallel Execution:} Utilize parallel backends (e.g., \texttt{furrr} or \texttt{future}) to distribute mutant executions across cores or nodes \cite{Vaughan2020furrr}.
  \item \textbf{Incremental Analysis:} Cache previous results and only re‐test affected mutants after code changes to avoid full-suite re‐execution on every run \cite{petrovic2018industrial}.
  \item \textbf{Resource Limits:} Allow users to configure timeouts and memory caps per mutant to prevent runaway tests and ensure predictable CI job durations.
\end{itemize}

\subsection{Scalability}
To handle large codebases and extensive test suites:
\begin{itemize}
  \item \textbf{Modular Architecture:} Separate mutation generation, test execution, and reporting into distinct components to facilitate horizontal scaling.
  \item \textbf{Streaming Results:} Emit partial reports as mutants are processed, enabling early feedback and integration with dashboards.
  \item \textbf{Batching:} Group mutants into batches to optimize test runner startup overhead and reduce per‐mutant initialization costs.
\end{itemize}

\subsection{Compatibility and Maintainability}
Ensuring the long‐term viability of the tool involves:
\begin{itemize}
  \item \textbf{R Version Support:} Test and certify compatibility with current and LTS releases of R and common package ecosystems (e.g., Bioconductor) \cite{gentleman2004bioconductor}.
  \item \textbf{Cross‐Platform Functionality:} Support Linux, Windows, and macOS environments with identical behavior.
  \item \textbf{Extensibility:} Provide plugin hooks for custom mutation operators and integration with other R-based tools (e.g., code coverage, linters).
\end{itemize}

\section{Application data flow}

%Data flow diagram showing where the data is coming from and where going %

%do we need this????% 
