% Solution design chapter % 

\begin{chapterabstract}
This chapter surveys the core technologies and constraints that underpin our R mutation‐testing tool. We first introduce the key libraries—devtools for streamlined package workflows, jsonlite for robust JSON handling, httr for HTTP communication, and furrr for parallel execution in R—alongside Google Test for C++ components and the OpenAI API for language‐model integration. We then outline the design constraints driving our architecture: customer and business requirements (seamless testthat integration, clear reporting, cost efficiency), performance and scalability (parallel execution, incremental analysis, batching), and compatibility/maintainability (cross‐platform support, modularity, extensibility). Together, these technologies and constraints form the foundation for our tool’s implementation and evaluation.
\end{chapterabstract}

\section{Technologies}

\subsection{Library \texttt{devtools}}

The \texttt{devtools} package streamlines R package development by providing functions to create package skeletons, load code, run tests, and build documentation in a single workflow \cite{wickham2019devtools}. It integrates tightly with \texttt{roxygen2} for in‐source documentation and with continuous integration services.

\begin{lstlisting}[
    float,
    caption={Creating, testing, and building an R package with \texttt{devtools}},
    label={lst:devtools-package-workflow},
    language=R,
    abovecaptionskip=-\medskipamount,
    belowcaptionskip=\medskipamount,
    basicstyle=\ttfamily\small,
    breaklines=true,
    captionpos=b
]
# Create a new package skeleton
library(devtools)
create("MyPackage")

# Load all R code without installing
load_all("MyPackage/")

# Run tests in tests/testthat/
test("MyPackage")

# Document and build package
document("MyPackage/")
build("MyPackage/")
\end{lstlisting}


By automating these repetitive tasks, \texttt{devtools} reduces boilerplate and accelerates iterative development.

\subsection{Library \texttt{jsonlite}}

The \texttt{jsonlite} package provides a robust, high‐performance JSON parser and generator for R, supporting both streaming and in‐memory operations. It maps R objects to JSON with consistent type coercion rules, making it ideal for API interaction and data interchange \cite{ooms2014jsonlite}.

\begin{lstlisting}[
    float,
    caption={Parsing and pretty-printing JSON with \texttt{jsonlite} in R},
    label={lst:jsonlite-parse-pretty},
    language=R,
    abovecaptionskip=-\medskipamount,
    belowcaptionskip=\medskipamount,
    basicstyle=\ttfamily\small,
    breaklines=true,
    captionpos=b
]
library(jsonlite)

# Parse JSON string into R list/data.frame
json_data <- fromJSON('{"x":1,"y":[2,3],"z":{"a":true}}')

# Convert R object back to pretty JSON
json_pretty <- toJSON(json_data, pretty = TRUE)
cat(json_pretty)
\end{lstlisting}


This two-way mapping ensures round-trip fidelity when exchanging data with web services.

\subsection{Library \texttt{httr}}

\texttt{httr} simplifies working with HTTP from R by wrapping low‐level HTTP methods in user-friendly functions. It handles authentication, redirection, and cookies, and parses responses into R objects \cite{wickham2011httr}.

\begin{lstlisting}[
    float,
    caption={Basic HTTP GET request and JSON parsing with \texttt{httr} in R},
    label={lst:httr-get-parse},
    language=R,
    abovecaptionskip=-\medskipamount,
    belowcaptionskip=\medskipamount,
    basicstyle=\ttfamily\small,
    breaklines=true,
    captionpos=b
]
library(httr)

# Basic GET request
resp <- GET("https://api.example.com/data",
            add_headers(Accept = "application/json"))

# Check for HTTP errors
stop_for_status(resp)

# Parse JSON content
data <- content(resp, as = "parsed")
\end{lstlisting}


For POST requests with JSON payloads, \texttt{httr} manages encoding and content‐type headers automatically.

\subsection{Library \texttt{furrr}}

The \texttt{furrr} package combines the mapping functions of \texttt{purrr} with the parallel execution capabilities of the \texttt{future} framework, allowing seamless parallel iteration in R \cite{Vaughan2020furrr}.

\begin{lstlisting}[
    float,
    caption={Parallel \texttt{future\_map} over a vector using \texttt{furrr}},
    label={lst:furrr-parallel-map},
    language=R,
    abovecaptionskip=-\medskipamount,
    belowcaptionskip=\medskipamount,
    basicstyle=\ttfamily\small,
    breaklines=true,
    captionpos=b
]
library(furrr)
plan(multisession, workers = 4)

# Parallel map over a vector
results <- future_map(1:10, function(x) {
  Sys.sleep(1)  # simulate work
  x^2
})
\end{lstlisting}


By changing the \texttt{plan()}, you can switch between multicore, multisession, or cluster‐based execution with minimal code changes.

\subsection{GoogleTest}

GoogleTest is Google’s C++ testing framework offering rich assertions, test fixtures, and death tests for verifying code correctness. Its expressive macros and automatic test discovery facilitate maintainable C++ test suites \cite{google2023gtest}.

\begin{lstlisting}[
    float,
    caption={Google Test example for the \texttt{square} function},
    label={lst:example-square-test},
    language=C++,
    abovecaptionskip=-\medskipamount,
    belowcaptionskip=\medskipamount,
    basicstyle=\ttfamily\small,
    breaklines=true,
    captionpos=b
]
// example.cpp
#include <gtest/gtest.h>

// Function under test
int square(int x) {
  return x * x;
}

// Test case
TEST(SquareTest, HandlesPositive) {
  EXPECT_EQ(square(3), 9);
  EXPECT_NE(square(3), 8);
}

int main(int argc, char **argv) {
  ::testing::InitGoogleTest(&argc, argv);
  return RUN_ALL_TESTS();
}
\end{lstlisting}


Compile and run with:
\begin{verbatim}
g++ example.cpp -lgtest -pthread -o example_test
./example_test
\end{verbatim}

\subsection{OpenAI API}

The OpenAI API provides RESTful endpoints for interacting with language models. Using \texttt{httr} in R, you can send requests and handle responses for completions, embeddings, and more \cite{openai2023api}.

\begin{lstlisting}[
    float,
    caption={POST request to OpenAI Completions API using \texttt{httr} and \texttt{jsonlite}},
    label={lst:openai-httr-post},
    language=R,
    abovecaptionskip=-\medskipamount,
    belowcaptionskip=\medskipamount,
    basicstyle=\ttfamily\small,
    breaklines=true,
    captionpos=b
]
library(httr)
library(jsonlite)

api_key <- Sys.getenv("OPENAI_API_KEY")

resp <- POST(
  "https://api.openai.com/v1/completions",
  add_headers(Authorization = paste("Bearer", api_key)),
  content_type_json(),
  encode = "json",
  body = list(
    model       = "text-davinci-003",
    prompt      = "Summarize the benefits of mutation testing:",
    max_tokens  = 100,
    temperature = 0.7
  )
)

stop_for_status(resp)
result <- content(resp, as = "parsed")
cat(result$choices[[1]]$text)
\end{lstlisting}


This integration enables programmatic access to advanced NLP capabilities directly from R.

\section{Design constraint}

Designing a mutation testing tool for the R language requires careful consideration of multiple constraints to ensure usability, maintainability, and performance in production environments.

\subsection{Customer Requirements}
The tool must cater to R users—ranging from data scientists to statisticians—by providing:
\begin{itemize}
  \item \textbf{Seamless Integration:} A familiar API that builds on \texttt{testthat} conventions, minimizing learning curve \cite{wickham2011testthat}.
  \item \textbf{Ease of Installation:} Simple installation via CRAN or \texttt{devtools::install\_github()}, with minimal external dependencies \cite{wickham2019devtools}.
  \item \textbf{Clear Reporting:} Human‐readable mutation score summaries and detailed mutant tracebacks for straightforward interpretation.
\end{itemize}

\subsection{Business Needs}
From an organizational standpoint, the tool must:
\begin{itemize}
  \item \textbf{Cost Efficiency:} Leverage existing infrastructure (R, CI/CD) to avoid additional licensing or hardware costs.
  \item \textbf{Return on Investment:} Provide actionable insights that improve test suite quality and reduce production defects, as shown by improved test robustness metrics in industry case studies \cite{petrovic2018industrial}.
  \item \textbf{Adoption and Support:} Comply with CRAN policies and support major R versions (≥3.6) to maximize user adoption and community contributions \cite{R-base}.
\end{itemize}

\subsection{Performance Constraints}
Mutation testing is inherently compute‐intensive, so the tool must:
\begin{itemize}
  \item \textbf{Parallel Execution:} Utilize parallel backends (e.g., \texttt{furrr} or \texttt{future}) to distribute mutant executions across cores or nodes \cite{Vaughan2020furrr}.
  \item \textbf{Incremental Analysis:} Cache previous results and only re‐test affected mutants after code changes to avoid full-suite re‐execution on every run \cite{petrovic2018industrial}.
  \item \textbf{Resource Limits:} Allow users to configure timeouts and memory caps per mutant to prevent runaway tests and ensure predictable CI job durations.
\end{itemize}

\subsection{Scalability}
To handle large codebases and extensive test suites:
\begin{itemize}
  \item \textbf{Modular Architecture:} Separate mutation generation, test execution, and reporting into distinct components to facilitate horizontal scaling.
  \item \textbf{Streaming Results:} Emit partial reports as mutants are processed, enabling early feedback and integration with dashboards.
  \item \textbf{Batching:} Group mutants into batches to optimize test runner startup overhead and reduce per‐mutant initialization costs.
\end{itemize}

\subsection{Compatibility and Maintainability}
Ensuring the long‐term viability of the tool involves:
\begin{itemize}
  \item \textbf{R Version Support:} Test and certify compatibility with current and LTS releases of R and common package ecosystems (e.g., Bioconductor) \cite{gentleman2004bioconductor}.
  \item \textbf{Cross‐Platform Functionality:} Support Linux, Windows, and macOS environments with identical behavior.
  \item \textbf{Extensibility:} Provide plugin hooks for custom mutation operators and integration with other R-based tools (e.g., code coverage, linters).
\end{itemize}
