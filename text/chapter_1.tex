\begin{chapterabstract}
% put abstract here %
\end{chapterabstract}

Mutation testing is a technique that involves deliberately introducing artificial errors into a software program to evaluate and improve the performance of a test suite. These modified versions of the program, known as 
\textit{mutants}, are generated by making slight, intentional changes to the original code. This approach not only assesses the robustness of existing test cases but also supports a variety of software quality assurance 
processes, such as prioritizing test cases, identifying bugs, and pinpointing fault locations. Mutation testing has evolved to become a high-performing method within contemporary testing and debugging practices.

R\cite{R-base} is an open-source programming language and environment primarily used for statistical analysis, data visualization, and computational research. Initially developed in the early 1990s by Ross Ihaka and Robert 
Gentleman, R has evolved into one of the most prominent languages in academia, data science, and bioinformatics. Its extensive ecosystem includes powerful packages for statistical modeling, machine learning, and graphical 
representation of data, making it highly favored by researchers and analysts who require advanced statistical capabilities alongside ease of visualization and reproducibility.

Several mutation testing frameworks have been developed specifically for R, providing tools to evaluate the quality of unit tests by introducing intentional faults (mutations) into the source code and checking if the test 
suite detects them.

One of the notable mutation testing frameworks for R is mutatr, which integrates directly with testthat\cite{wickham2011testthat}, the most commonly used testing framework for R projects. Mutatr\cite{wickham_mutatr} 
generates mutations by systematically altering R code elements like conditional operators, mathematical operations, and function calls. Despite its seamless integration, mutatr has some limitations, such as limited 
configurability and flexibility. It can become computationally expensive for larger projects, as it does not efficiently manage mutation selection or parallel execution, thus slowing down the development feedback loop.

This paper presents the development and evaluation of a mutation testing tool for the R programming language. By introducing small, controlled changes into R code, the tool assesses the effectiveness of test suites in 
detecting faults. The study outlines the tool’s implementation and demonstrates its potential to improve the reliability of R-based software.