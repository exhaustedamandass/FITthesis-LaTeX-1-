%Solution implementation%

%put some pseudocode here, maybe a class diagram%

\begin{chapterabstract}
This chapter details the implementation of the mutation testing tool, combining native C++ code with a high-level R interface. We begin by presenting the C++ core—\texttt{C\_mutate\_file} and \texttt{C\_mutate\_single}—which leverage R’s \texttt{SEXP} API to traverse ASTs (via \texttt{ASTHandler}) and apply operator‐flip and delete mutations (via \texttt{Mutator}), supported by a modular operator class hierarchy for arithmetic, logical, unary, and delete operations \cite{R-base,wickham_pairlists}. Next, we describe the R side: \texttt{mutate\_package} orchestrates per‐file mutations (\texttt{mutate\_file}), which in turn calls the C routines and falls back on \texttt{delete\_line\_mutants} for string‐deletion mutations. Surviving mutants are evaluated in parallel with \texttt{testthat} and \texttt{furrr} for efficient test execution \cite{wickham2011testthat,Vaughan2020furrr}. Finally, semantic equivalence detection is performed by \texttt{identify\_equivalent\_mutants} using the OpenAI API to filter out equivalent mutants before computing raw and adjusted mutation scores \cite{wickham2011httr,ooms2014jsonlite,openai2023api}.

\end{chapterabstract}

\section{C++ Implementation}

In this chapter, we detail the core C++ routines that drive the mutation testing engine’s integration with R’s internal API. We focus on two central functions, \texttt{C\_mutate\_file} and \texttt{C\_mutate\_single}, which together enumerate, apply, validate, and collect mutants for an entire R source file or a single expression, respectively. These functions rely on R’s \texttt{SEXP} type, protection stack, and evaluation APIs to safely manipulate R objects in native code \cite{R-base}.

\subsection{\texttt{C\_mutate\_file} Function}

The \texttt{C\_mutate\_file} function orchestrates mutation across all expressions in an R source block.  Its high-level pseudocode is:

\begin{lstlisting}[
    float,
    caption={Processing entire file expressions for mutations in `C\_mutate\_file`},
    label={lst:c-mutate-file},
    language=C,
    abovecaptionskip=-\medskipamount,
    belowcaptionskip=\medskipamount,
    basicstyle=\ttfamily\small,
    breaklines=true,
    captionpos=b
]
// C_mutate_file(exprs):
//   1. Validate input is EXPRSXP list with a matching 'srcref' attribute.
//   2. Determine which expressions lie inside `{ }` blocks.
//   3. For each expression i:
//        a. Call C_mutate_single(expr_i, src_ref_i, inside_block[i]) 
//           → list of per-expression mutants.
//        b. For each mutant j:
//             • Build a full-file mutant by replacing expr_i with mutant_j.
//             • Copy 'mutation_info' from mutant_j into the file mutant.
//             • Protect the new file mutant on the R protection stack.
//             • If R_tryEval(file_mutant) succeeds, keep it; otherwise discard.
//   4. Allocate and return an R list containing all valid, protected mutants.
\end{lstlisting}


Key implementation notes:
\begin{itemize}
  \item Uses \texttt{detect\_block\_expressions} to mark expressions inside code blocks.
  \item Manages protection via \texttt{PROTECT}/\texttt{UNPROTECT} and accumulates valid mutants in a C++ \texttt{std::vector<SEXP>}.
  \item Leverages \texttt{R\_tryEval} to filter out syntactically or semantically invalid mutants.
\end{itemize}

\subsection{\texttt{C\_mutate\_single} Function}

The \texttt{C\_mutate\_single} function focuses on a single R expression, gathering mutation points and applying each operator to produce candidate mutants.  The core pseudocode is:

\begin{lstlisting}[
    float,
    caption={Workflow of single‐expression mutation in `C\_mutate\_single`},
    label={lst:c-mutate-single},
    language=C,
    abovecaptionskip=-\medskipamount,
    belowcaptionskip=\medskipamount,
    basicstyle=\ttfamily\small,
    breaklines=true,
    captionpos=b
]
// C_mutate_single(expr, src_ref, is_inside_block):
//   1. If expr is an EXPRSXP of length 0 → error.
//   2. If expr is EXPRSXP with multiple elements, extract the first.
//   3. astHandler.gatherOperators(expr, src_ref, is_inside_block)
//        → vector of (location, operator) pairs.
//   4. If no operators → return empty list.
//   5. For each operator index i:
//        a. (mut_expr, ok) = mutator.applyMutation(expr, operators, i)
//        b. If ok:
//             • PROTECT(mut_expr); add to mutants vector.
//   6. Allocate VECSXP list of size = mutants.size(); protect it.
//   7. Populate result list with each protected mutant.
//   8. UNPROTECT all but the result; return the result list.
\end{lstlisting}


Important details:
\begin{itemize}
  \item Handles both raw expressions (\texttt{LANGSXP}, \texttt{SYMSXP}, etc.) and expression vectors (\texttt{EXPRSXP}).
  \item Calls into \texttt{ASTHandler} to locate mutation sites and \texttt{Mutator} to generate concrete mutants.
  \item Carefully tracks the number of \texttt{PROTECT} calls to avoid garbage–collection issues.
\end{itemize}


\subsection{ASTHandler class}

The \texttt{ASTHandler} class is responsible for scanning an R expression’s abstract syntax tree (AST), identifying all mutation points (operators and deletable sub‐expressions), and recording their source locations.  It uses cached R symbols for efficient comparison and a recursive traversal to collect operator positions and delete‐candidates in a structured \texttt{OperatorPos} vector.

\begin{lstlisting}[
    float,
    caption={Cached R symbols for operators and source reference tags},
    label={lst:cached-symbols},
    language=C++,
    abovecaptionskip=-\medskipamount,
    belowcaptionskip=\medskipamount,
    basicstyle=\ttfamily\small,
    breaklines=true,
    captionpos=b
]
// Cached R symbols for operators and srcref tags
static struct CachedSyms {
  SEXP s_lbrace   = Rf_install("{");
  SEXP s_rbrace   = Rf_install("}");
  SEXP s_plus     = Rf_install("+");
  SEXP s_minus    = Rf_install("-");
  // … other operators: *, /, ==, !=, <, >, <=, >=, &, |, &&, ||
  SEXP s_srcref   = Rf_install("srcref");
  SEXP s_mutinfo  = Rf_install("mutation_info");
} SYM;
\end{lstlisting}


\noindent\textbf{Symbol Caching}: Installing symbols once at startup avoids repeated lookups during traversal, improving performance when processing many expressions \cite{R-base}.

\medskip
\noindent\textbf{Deletable Check}  
Determines whether an expression can be safely removed (for a “delete” mutation) only when inside a block, excluding the block delimiters themselves:

\begin{lstlisting}[
    float,
    caption={Determine if an expression can be deleted within a block},
    label={lst:is-deletable},
    language=C++,
    abovecaptionskip=-\medskipamount,
    belowcaptionskip=\medskipamount,
    basicstyle=\ttfamily\small,
    breaklines=true,
    captionpos=b
]
// isDeletable(expr):
//   if not inside a { } block → return false
//   if expr is a call to “{” or “}” → return false
//   otherwise → return true
bool isDeletable(SEXP expr) {
  if (!_is_inside_block) return false;
  if (TYPEOF(expr) == LANGSXP) {
    SEXP head = CAR(expr);
    if (head == SYM.s_lbrace || head == SYM.s_rbrace)
      return false;
  }
  return true;
}
\end{lstlisting}


\medskip
\noindent\textbf{Operator Gathering Entry Point}  
Initializes source reference bounds and block context, then kicks off the recursive walk:

\begin{lstlisting}[
    float,
    caption={Gather operators with source reference and block flag},
    label={lst:gather-operators-setup},
    language=C++,
    abovecaptionskip=-\medskipamount,
    belowcaptionskip=\medskipamount,
    basicstyle=\ttfamily\small,
    breaklines=true,
    captionpos=b
]
// gatherOperators(expr, src_ref, inside_block):
//   validate src_ref is integer[4]
//   set source location (_start_line, _start_col, _end_line, _end_col)
//   set _is_inside_block flag
//   call gatherOperatorsRecursive(expr, path = [], ops = [])
std::vector<OperatorPos> gatherOperators(
    SEXP expr, SEXP src_ref, bool inside_block) {
  // … validate src_ref …
  _is_inside_block = inside_block;
  std::vector<OperatorPos> ops;
  gatherOperatorsRecursive(expr, {}, ops);
  return ops;
}
\end{lstlisting}


\medskip
\noindent\textbf{Recursive Traversal}  
Walks the AST in depth‐first order.  At each call node, it:
1. Checks if the function symbol matches a known operator in \texttt{op\_map} and, if so, creates the corresponding \texttt{Operator} instance.  
2. Adds a \texttt{DeleteOperator} if \texttt{isDeletable(expr)}.  
3. Recurses on each argument sub‐expression, tracking the path indices for precise mutation placement.

\begin{lstlisting}[
    float,
    caption={Recursive gathering of mutation operators in the AST},
    label={lst:gather-operators},
    language=C++,
    abovecaptionskip=-\medskipamount,
    belowcaptionskip=\medskipamount,
    basicstyle=\ttfamily\scriptsize,
    breaklines=true,
    captionpos=b
]
// gatherOperatorsRecursive(expr, path, ops):
//   if expr not a LANGSXP → return
//   fun = CAR(expr)
//   if fun in op_map: record OperatorPos(...)
void gatherOperatorsRecursive(
    SEXP expr,
    std::vector<int> path,
    std::vector<OperatorPos>& ops
) {
  if (TYPEOF(expr) != LANGSXP) return;
  SEXP fun = CAR(expr);

  static const std::map<SEXP,
    std::function<std::unique_ptr<Operator>()>> op_map = {
      {SYM.s_plus,  []{ return std::make_unique<PlusOperator>(); }},
      {SYM.s_minus, []{ return std::make_unique<MinusOperator>(); }},
      // … other operators …
  };
  if (auto it = op_map.find(fun); it != op_map.end()) {
    ops.push_back({path, it->second(), _start_line, _start_col,
                   _end_line, _end_col, fun});
  }

  if (isDeletable(expr)) {
    ops.push_back({path,
                   std::make_unique<DeleteOperator>(expr),
                   _start_line, _start_col,
                   _end_line, _end_col, expr});
  }

  int idx = 0;
  for (SEXP arg = CDR(expr); arg != R_NilValue; arg = CDR(arg), ++idx) {
    auto new_path = path; new_path.push_back(idx);
    gatherOperatorsRecursive(CAR(arg), new_path, ops);
  }
}
\end{lstlisting}


This design cleanly separates symbol lookup, context checking, and AST traversal, producing a comprehensive list of mutation candidates that downstream code can transform and validate \cite{R-base}.

\subsection{Mutator class}

The \texttt{Mutator} class is responsible for taking a single R expression and a list of identified mutation sites (\texttt{OperatorPos} objects), then producing mutated copies without altering the original code. It cleanly separates two mutation strategies—flipping operators (e.g., “+” to “–”) and deleting sub-expressions—and tags each mutant with human-readable metadata indicating where and how the mutation occurred.

\medskip
\noindent\textbf{Dispatch Logic}  
At entry, \texttt{applyMutation} simply checks bounds and delegates based on the operator type:

\begin{lstlisting}[
    float,
    caption={Range check and dispatch to delete or flip mutation},
    label={lst:dispatch-mutation},
    language=C,
    abovecaptionskip=-\medskipamount,
    belowcaptionskip=\medskipamount,
    basicstyle=\ttfamily\small,
    breaklines=true,
    captionpos=b
]
// if out of range → fail
// if operator is Delete → applyDeleteMutation
// else → applyFlipMutation
\end{lstlisting}


\medskip
\noindent\textbf{Flip Mutation}  
Creates a protected duplicate, navigates to the target node via its path, applies the operator’s \texttt{flip()}, and annotates with a \texttt{mutation\_info} string:

\begin{lstlisting}[
    float,
    caption={Navigating to node, applying mutation operator, and annotating info},
    label={lst:apply-mutation-steps},
    language=C,
    abovecaptionskip=-\medskipamount,
    belowcaptionskip=\medskipamount,
    basicstyle=\ttfamily\small,
    breaklines=true,
    captionpos=b
]
// PROTECT(dup = duplicate(expr))
// navigate by pos.path → node
// pos.op->flip(node)
// attach mkString(info) as "mutation_info"
\end{lstlisting}


\medskip
\noindent\textbf{Delete Mutation}  
Duplicates the expression, finds the parent CONS cell of the target, updates its \texttt{CDR} pointer to skip the child, and records deletion details:

\begin{lstlisting}[
    float,
    caption={Duplication, deletion, and annotation steps in mutation logic},
    label={lst:mutation-comments},
    language=C,
    abovecaptionskip=-\medskipamount,
    belowcaptionskip=\medskipamount,
    basicstyle=\ttfamily\small,
    breaklines=true,
    captionpos=b
]
// PROTECT(dup = duplicate(expr))
// find parent via path[:-1]
// SETCDR(prev, CDDR(prev))  // remove node
// attach mkString(info) as "mutation_info"
\end{lstlisting}


All PROTECT calls are balanced with appropriate UNPROTECTs so that only the final mutated object remains on R’s protection stack, ensuring memory safety when returning to R’s garbage collector.  


\subsection{Operators}

MutatoR’s mutation operators are organized into a clear class hierarchy rooted in a common \texttt{Operator} interface. This design ensures that new operator types can be added with minimal boilerplate, while shared logic (e.g.\ symbol management, metadata attachment) lives in base classes. All operators derive from:

\begin{lstlisting}[
    float,
    caption={Abstract `Operator` interface for mutation classes},
    label={lst:operator-interface},
    language=C++,
    abovecaptionskip=-\medskipamount,
    belowcaptionskip=\medskipamount,
    basicstyle=\ttfamily\small,
    breaklines=true,
    captionpos=b
]
class Operator {
  SEXP operator_symbol;
  virtual void flip(SEXP& node) const = 0;
  virtual std::string getType() const = 0;
};
\end{lstlisting}


\noindent From there, operators split into several families:

\subsubsection{Arithmetic and Logical Operators}

These operators inherit from \texttt{ArithmeticOperator} or \texttt{LogicalOperator}, which themselves extend \texttt{Operator}:

\begin{lstlisting}[
    float,
    caption={Arithmetic mutation operators: base class and division/multiplication swaps},
    label={lst:arithmetic-operators},
    language=C++,
    abovecaptionskip=-\medskipamount,
    belowcaptionskip=\medskipamount,
    basicstyle=\ttfamily\small,
    breaklines=true,
    captionpos=b
]
// Base for +, -, *, /
class ArithmeticOperator : public Operator {
  …
};

// Swap * → /
class DivideOperator : public ArithmeticOperator {
  void flip(SEXP& node) const override {
    SETCAR(node, Rf_install("*"));
  }
};

// Swap / → *
class MultiplyOperator : public ArithmeticOperator {
  void flip(SEXP& node) const override {
    SETCAR(node, Rf_install("/"));
  }
};
\end{lstlisting}


Logical operators follow the same pattern (e.g.\ \texttt{AndOperator} flips “\&” to “|”, \texttt{LogicalOrOperator} flips “||” to “\&\&”) \cite{offutt1996practical,jia2011analysis}.

\subsubsection{Unary Operator Mutations}

Unary operators remove or invert single-operand expressions. They subclass a lightweight \texttt{UnaryOperator} base:

\begin{lstlisting}[
    float,
    caption={RemoveNegation class for dropping logical negation},
    label={lst:remove-negation},
    language=C++,
    abovecaptionskip=-\medskipamount,
    belowcaptionskip=\medskipamount,
    basicstyle=\ttfamily\small,
    breaklines=true,
    captionpos=b
]
// Remove negation: !x → x
class RemoveNegation : public Operator {
  void flip(SEXP& node) const override {
    // drop the '!' by replacing the call with its argument
    node = CAR(CDR(node));
  }
};
\end{lstlisting}


\subsubsection{Delete Operator}

Beyond flips, the \texttt{DeleteOperator} removes entire sub-expressions when they occur inside a block. It does not subclass a more specific family, but directly extends \texttt{Operator}:

\begin{lstlisting}[
    float,
    caption={DeleteOperator class for rewiring parent CDR pointer},
    label={lst:delete-operator},
    language=C++,
    abovecaptionskip=-\medskipamount,
    belowcaptionskip=\medskipamount,
    basicstyle=\ttfamily\small,
    breaklines=true,
    captionpos=b
]
// Delete a node by rewiring its parent CDR pointer
class DeleteOperator : public Operator {
  void flip(SEXP& node) const override {
    // actual deletion logic happens in Mutator::applyDeleteMutation
  }
};
\end{lstlisting}


\subsubsection{Extensibility}

Adding a new operator involves:

\begin{enumerate}
  \item Creating a subclass of \texttt{Operator} (or an existing family) with a constructor installing the target symbol.
  \item Implementing \texttt{flip(SEXP\&)} to perform the AST transformation.
  \item Registering the operator in \texttt{ASTHandler}’s \texttt{op\_map} to enable its detection.
\end{enumerate}

This modular hierarchy balances code reuse with the flexibility to support any mutation pattern required for thorough test‐suite evaluation.

\section{R implementation}

In addition to the low‐level C++ routines, the mutation testing tool provides a high‐level R interface that orchestrates mutant generation, test execution, and equivalence analysis entirely within R.  This section describes the four main R functions that drive the workflow: \texttt{mutate\_package}, \texttt{mutate\_file}, \texttt{identify\_equivalent\_mutants}, and \texttt{delete\_line\_mutants}.

\subsection{Mutate Package Function}

The \texttt{mutate\_package} function is the user’s entry point for mutating an entire R package.  It performs these steps:

\begin{enumerate}
  \item Locate all R source files under \texttt{pkg\_dir/R/}.  
  \item For each source file, call \texttt{mutate\_file()} to produce a list of mutants (AST‐based and line‐deletion).  
  \item For each mutant, copy the package to a temporary directory, replace the original file with the mutant, and record the mutation path and metadata.  
  \item Run \texttt{testthat} tests in parallel across mutants using \texttt{furrr} to determine which mutants survive \cite{wickham2011testthat,Vaughan2020furrr}.  
  \item Optionally, invoke \texttt{identify\_equivalent\_mutants()} on the survivors.  
  \item Summarize total, killed, survived, and (if requested) equivalent mutants, computing both raw and adjusted mutation scores.
\end{enumerate}

\medskip
\begin{lstlisting}[
    float,
    caption={Mutating an entire package and running tests in parallel},
    label={lst:mutate-package},
    language=R,
    abovecaptionskip=-\medskipamount,
    belowcaptionskip=\medskipamount,
    basicstyle=\ttfamily\small,
    breaklines=true,
    captionpos=b
]
mutate_package <- function(pkg_dir,
                           cores = parallel::detectCores(),
                           ...) {
  r_files <- list.files(
    file.path(pkg_dir, "R"),
    "\\.R$",
    full.names = TRUE
  )
  mutants <- map(
    r_files,
    ~ mutate_file(.x, out_dir = tmp)
  )
  # copy package, replace file, run tests in parallel
  results <- future_map(
    mutants,
    run_tests,
    .progress = TRUE
  )
  # collate and summarize
  ...
}
\end{lstlisting}


\subsection{Mutate File Function}

The \texttt{mutate\_file} function handles a single R source file.  Its workflow:

\begin{enumerate}
  \item Ensure output directory exists and enable \texttt{keep.source=TRUE} so that \texttt{srcref} attributes are preserved.  
  \item Parse the file via \texttt{parse(..., keep.source=TRUE)} and attach default \texttt{srcref} if missing.  
  \item Call the C routine \texttt{.Call("C\_mutate\_file", parsed)} to generate AST‐based mutants.  
  \item Deparse each mutant expression back to text, write to a new file under \texttt{out\_dir}, and collect its path and \texttt{mutation\_info} attribute.  
  \item Append up to \texttt{max\_del} string‐deletion mutants via \texttt{delete\_line\_mutants()} as a fallback.
\end{enumerate}

\medskip
\noindent\textbf{Key snippet:}
\begin{lstlisting}[
    float,
    caption={Generating and writing mutated versions of the parsed code},
    label={lst:write-mutations},
    language=R,
    abovecaptionskip=-\medskipamount,
    belowcaptionskip=\medskipamount,
    basicstyle=\ttfamily\small,
    breaklines=true,
    captionpos=b
]
parsed <- parse(src_file, keep.source = TRUE)
raw    <- .Call("C_mutate_file", parsed)
for (m in raw) {
  code <- paste(deparse(m), collapse = "\n")
  writeLines(code, out_file)
  results[[i]] <- list(
    path = out_file,
    info = attr(m, "mutation_info")
  )
}
# string‐deletion fallback
results <- c(
  results,
  delete_line_mutants(src_file, out_dir, max_del = 5)
)
\end{lstlisting}


\subsection{Identify Equivalent Mutants Function}

To detect equivalent mutants—those that survive tests yet behave identically—the \texttt{identify\_equivalent\_mutants} function leverages the OpenAI API:

\begin{enumerate}
  \item Load API credentials via \texttt{get\_openai\_config()} or environment variables.  
  \item Read the original source file into a single string.  
  \item Group surviving mutants by file, extracting their IDs and \texttt{mutation\_info}.  
  \item Construct a formatted prompt with the original code and each mutant’s details using \texttt{create\_equivalent\_mutant\_prompt()}.  
  \item Send the prompt to OpenAI via \texttt{httr::POST()} and \texttt{jsonlite::toJSON()} \cite{wickham2011httr,ooms2014jsonlite,openai2023api}.  
  \item Parse the response to set an \texttt{equivalent} flag for each mutant based on simple pattern matching.
\end{enumerate}

\medskip
\noindent\textbf{Key snippet:}
\begin{verbatim}
prompt <- create_equivalent_mutant_prompt(orig_code, mutant_details)
resp   <- call_openai_api(prompt, api_config)
if (!is.null(resp$choices)) {
  analysis <- resp$choices[[1]]$message$content
  survived_mutants[[mid]]$equivalent <- grepl("equivalent", analysis, ignore.case=TRUE)
}
\end{verbatim}

\subsection{Delete Line Mutants Function}

As a complementary strategy, \texttt{delete\_line\_mutants} generates “string‐deletion” mutants by removing individual non‐empty lines:

\begin{enumerate}
  \item Read all lines from \texttt{src\_file} and identify non‐empty indices.  
  \item For up to \texttt{max\_del} iterations, randomly pick a non‐empty line, remove it, and write the result to a new file in \texttt{out\_dir}.  
  \item Return a list of mutant descriptors with file paths and a deletion message.
\end{enumerate}

\medskip
\noindent\textbf{Key snippet:}
\begin{lstlisting}[
    float,
    caption={Deleting one random non‐empty line from the source file},
    label={lst:delete-random-line}
]
lines     <- readLines(src_file)
non_empty <- which(nzchar(lines))
idx       <- sample(non_empty, 1)
writeLines(lines[-idx], out_file)
mutants[[i]] <- list(
  path = out_file,
  info = sprintf("deleted line %d", idx)
)
\end{lstlisting}


Together, these R functions enable an end‐to‐end mutation testing workflow: from AST‐driven and line‐deletion mutant generation, through parallel test execution and equivalence analysis, to final mutation score reporting.